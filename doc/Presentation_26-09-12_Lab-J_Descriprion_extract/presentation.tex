\documentclass[slidestop]{beamer}

\title{Extracting HGVS descriptions}
\providecommand{\myConference}{Work discussion}
\providecommand{\myDate}{Thursday, 24 February 2011}
\author{Jeroen F. J. Laros}
\providecommand{\myGroup}{Leiden Genome Technology Center}
\providecommand{\myDepartment}{Department of Human Genetics}
\providecommand{\myCenter}{Center for Human and Clinical Genetics}
\providecommand{\lastCenterLogo}{
  \raisebox{-0.1cm}{
    \includegraphics[height = 1cm]{lgtc_logo}
    %\includegraphics[height = 0.7cm]{ngi_logo}
  }
}
\providecommand{\lastRightLogo}{
  %\includegraphics[height = 0.7cm]{nbic_logo}
  %\includegraphics[height = 0.8cm]{nwo_logo_en}
  \hspace{1.5cm}\includegraphics[height = 0.7cm]{gen2phen_logo}
}

\usetheme{lumc}

\usepackage{ifthen}

../Presentation_24-02-11_HumGen_Mutalyzer2/lstBNF.tex

\begin{document}

\newcommand{\algorithmexample}[1]{
  \begin{figure}[]
    \begin{center}
      \fbox{
        \setlength{\unitlength}{1pt}
        \linethickness{3pt}
        \begin{picture}(300, 60)(0, 0)
          \put(0, 10){\line(1, 0){30}} % Observed sequence.
          \put(30, 10){\color{red}\line(1, 0){240}\color{white}} % Change.
          \put(270, 10){\line(1, 0){30}}
          \put(0, 14){{\scriptsize observed}}

          \put(0, 40){\line(1, 0){30}} % Reference sequence.
          \put(30, 40){\color{green}\line(1, 0){240}\color{white}} % Change.
          \put(270, 40){\line(1, 0){30}}
          \put(0, 46){{\scriptsize reference}}
          \put(30, 30){{\scriptsize $8$}}
          \put(270, 30){{\scriptsize $98$}}

          \ifthenelse{\equal{#1}{1}}{
            \drawcurve(50, 40)(55, 35)(155, 25)(255, 15)(260, 10)
            \drawcurve(260, 40)(255, 35)(155, 25)(55, 15)(50, 10)
          }{}
          \ifthenelse{#1>1}{
            \put(50, 10){\line(1, 0){210}} % Inv.
            \put(50, 40){\line(1, 0){210}} % Inv.
          }{}
          \ifthenelse{#1>2}{
            \put(35, 10){\line(1, 0){10}}
            \put(35, 40){\line(1, 0){10}}
          }{}
        \end{picture}
      }
    \end{center}
    \caption{How would a human do it?}
  \end{figure}
}

% This disables the \pause command, handy in the editing phase.
%\renewcommand{\pause}{}

% Make the title page.
\bodytemplate

% First page of the presentation.
\section{Introduction}

\begin{frame}
  \frametitle{Mutalyzer}
  
  A curational tool for \emph{Locus Specific Mutation Databases} (LSDBs).
  \pause
  \bigskip

  Variant nomenclature checker applying \emph{Human Genome Variation Society}
  (HGVS) guidelines.
  \begin{itemize}
    \item Is the syntax of the variant description valid?
    \item Does the reference sequence exist?
    \item Is the variant possible on this reference sequence?
    \item Is this variant description the recommended one?
  \end{itemize}
  \bigskip
  \pause

  Basic effect prediction.
  \begin{itemize}
    \item Is the description of the transcript product as expected?
    \item Is the predicted protein as expected?
  \end{itemize}
\end{frame}

\begin{frame}
  \frametitle{Mutalyzer}

  Nowadays Mutalyzer is a vital part of LOVD version 3.
  \bigskip
  \pause

  Make a reference sequence (configure new gene):
  \begin{itemize}
    \item Given a gene symbol, make a slice of a chromosome.
    \item Receive information on the transcripts and genes in a genomic
      reference sequence.
  \end{itemize}
  \medskip
  \pause

  Mapping variants:
  \begin{itemize}
    \item Find which transcript is affected.
    \item Map variants to the genome and vice versa.
    \item Lift a description over to an other transcript.
  \end{itemize}
  \medskip
  \pause

  Curating submissions:
  \begin{itemize}
    \item Checking the syntax.
    \item Checking the variant description.
  \end{itemize}
\end{frame}

\section{HGVS nomenclature}
\begin{frame}
  \frametitle{HGVS descriptions}

  A simple variant:

  \bt{NM\_002001.2:c.25A>T}
  \bigskip
  \pause

  \begin{table}[]
    \begin{center}
      \begin{tabular}{c|l}
        Token             & meaning \\
        \hline
        \bt{NM\_002001.2} & Reference sequence and version. \\
        \bt{c.}           & Coordinate system. \\
        \bt{25}           & Position within a coordinate system. \\
        \bt{A>T}          & Variant (substitution). \\
      \end{tabular}
    \end{center}
    \caption{A simple variant description.}
  \end{table}
  \bigskip
  \pause

  Combine simple variants to complex ones:

  \bt{NM\_002001.2:c.[25A>T;100del]}
\end{frame}

\begin{frame}[fragile]
  \frametitle{HGVS syntax}
  \pause

  Definition of a gene symbol.
  \begin{lstlisting}[language = BNF, caption = {Abstract HGVS nomenclature}]
    TransVar   -> `_v' Number
    ProtIso    -> `_i' Number
    GeneSymbol -> `(' Name (TransVar | ProtIso)? `)'
  \end{lstlisting}
  \bigskip
  \pause

  Gene name and optionally a transcript or isoform number.

  \begin{lstlisting}[caption = {HGVS nomenclature in Python}]
      TransVar = Suppress("_v") + Number("TransVar")
      ProtIso = Suppress("_i") + Number("ProtIso")
      GeneSymbol = Suppress('(') + \
          Group(Name("GeneSymbol") + \
          Optional(TransVar ^ ProtIso))("Gene") + \
          Suppress(')')
  \end{lstlisting}
\end{frame}

\begin{frame}
  \frametitle{HGVS semantics}

  There are a few guidelines for describing variants:
  \begin{itemize}
    \item Always use the most 3' variant description.
    \item Use the shortest description.
  \end{itemize}
  \bigskip
  \pause

  There are no guidelines on \emph{how} to do this.
  \bigskip

  Example: we observe a change from \bt{CCCCCCC} to \bt{CACACAC}.
  \begin{itemize}
    \item \bt{2\_6\color{yellow}delins\color{white}ACACA}
    \item \bt{[2C\color{yellow}>\color{white}A;4C\color{yellow}>\color{white}A;6C\color{yellow}>\color{white}A]}
    \item \bt{[1\_2\color{yellow}ins\color{white}A;3\_6\color{yellow}delins\color{white}ACA]}
    \item \ldots
  \end{itemize}
\end{frame}

\begin{frame}
  \frametitle{Variants are not ``inherited''}

  Silent mutations for example.
  \bigskip
  \pause

  A double frameshift:

  \bt{NM\_002001.2:c.[10del;22\_23del]}

  \bt{NP\_001992.1:p.Ala4\_Pro7delinsProTrpAsn}
  \bigskip
  \pause

  A complex variant that leads to a simple protein change:

  \bt{NM\_002001.2:[c.10\_12delinsAAA;102G>A]}

  \bt{NP\_001992.1:p.Ala4Lys}
  \bigskip
  \pause

  An insertion that affects two codons:

  \bt{NM\_002001.2:c.10\_11insTTT}

  \bt{NP\_001992.1:p.Ala4delinsValSer}
\end{frame}

\begin{frame}
  \frametitle{Problem description}

  Verifying the validity of a variant description is not enough:
  \begin{itemize}
    \item Both \bt{5\_7delinsATA} and \bt{[6G>A;7C>A]} are valid.
    \item We want one representation.
  \end{itemize}
  \bigskip
  \pause

  We need something that:
  \begin{itemize}
    \item Accepts any description to modify a reference sequence.
    \item Compares the reference and the modified sequence to make a
      description.
  \end{itemize}
  \bigskip
  \pause

  A description extractor.
\end{frame}

\section{Extracting descriptions}
\begin{frame}
  \frametitle{A ``human'' way of finding a description}

  Observation:
  \begin{itemize}
    \item There is always a default way of describing a variant (\bt{delins}).
    \item A \bt{delins} may be split in smaller parts.
  \end{itemize}
  \bigskip
  \pause

  Outline:
  \begin{itemize}
    \item Find the \emph{area of change}.
    \item Describe this as a \bt{delins}.
    \item Find the largest overlap in this area of change, splitting the area
      in two.
    \item Describe the two sub areas, and see whether this description is
      smaller than the one we have.
  \end{itemize}
\end{frame}

\begin{fframe}
  \frametitle{Outline of the algorithm}

  \only<1>{\algorithmexample{0}}
  \only<2>{\algorithmexample{1}}
  \only<3>{\algorithmexample{2}}
  \only<4>{\algorithmexample{3}}

  \bt{8\_98\color{yellow}delins\color{white}AGATGCGATAGATTAGCTATATAGGATCG\ldots}
  \onslide<3->{\bt{[8\_12\color{yellow}delins\color{white}AGATG;13\_96\color{yellow}inv\color{white};97\_98\color{yellow}delins\color{white}TG]}}

  \onslide<4->{\bt{[8G\color{yellow}>\color{white}A;12C\color{yellow}>\color{white}G;13\_96\color{yellow}inv\color{white};97\_98\color{yellow}delins\color{white}TG]}}

  \vfill
\end{fframe}

\begin{fframe}
  \frametitle{Finding common sub strings}

  How would a computer do it?
  \begin{table}[]
    \begin{center}
      \begin{tabular}{l|lllllll}
          & \bt{A} & \bt{T} & \bt{G} & \bt{A} & \bt{G} & \bt{C} & \bt{G} \\
        \hline
        \bt{A} & \onslide<2>{\color{red}}1 & 0 & 0 &
          \onslide<3>{\color{gray}}1 & \onslide<3>{\color{gray}}0 &
          \onslide<3>{\color{gray}}0 & \onslide<3>{\color{gray}}0 \\
        \bt{T} & 0 & \onslide<2>{\color{red}}2 & 0 &
          \onslide<3>{\color{gray}}0 & \onslide<3>{\color{gray}}0 &
          \onslide<3>{\color{gray}}0 & \onslide<3>{\color{gray}}0 \\
        \bt{C} & 0 & 0 & 0 & \onslide<3>{\color{gray}}0 &
          \onslide<3>{\color{gray}}0 & \onslide<3>{\color{gray}}1 &
          \onslide<3>{\color{gray}}0 \\
        \bt{A} & \onslide<3>{\color{gray}}1 & \onslide<3>{\color{gray}}0 &
          \onslide<3>{\color{gray}}0 & \onslide<3>{\color{gray}}1 &
          \onslide<3>{\color{gray}}0 & \onslide<3>{\color{gray}}0 &
          \onslide<3>{\color{gray}}0 \\
        \bt{G} & \onslide<3>{\color{gray}}0 & \onslide<3>{\color{gray}}0 &
          \onslide<3>{\color{gray}}1 & \onslide<3>{\color{gray}}0 &
          \onslide<3>{\color{gray}}2 & \onslide<3>{\color{gray}}0 &
          \onslide<3>{\color{gray}}1 \\
        \bt{C} & \onslide<3>{\color{gray}}0 & \onslide<3>{\color{gray}}0 &
          \onslide<3>{\color{gray}}0 & \onslide<3>{\color{gray}}0 &
          \onslide<3>{\color{gray}}0 & \onslide<3>{\color{gray}}3 &
          \onslide<3>{\color{gray}}0 \\
        \bt{A} & \onslide<3>{\color{gray}}1 & \onslide<3>{\color{gray}}0 &
          \onslide<3>{\color{gray}}0 & \onslide<3>{\color{gray}}1 &
          \onslide<3>{\color{gray}}0 & \onslide<3>{\color{gray}}0 & 0 \\
      \end{tabular}
    \end{center}
    \caption{LCS dynamic programming.}
  \end{table}

  \only<2>{Reusing partial solutions.}
  \only<3>{Reusing parts of the matrix.}

  \vfill
\end{fframe}

\section{Results}
\begin{frame}
  \frametitle{Protein descriptions}

  Input:

  \bt{NM\_002001.2:n.[109G>T;139G>T;159del]}
  \bigskip
  \pause

  Old:

  \bt{NM\_002001.2:n.[109G>T;139G>T;159del]}

  \bt{NM\_002001.2:p.?}
  \bigskip
  \pause

  New:

  \bt{NM\_002001.2:n.[109G>T;139G>T;159del]}

  \bt{NM\_002001.2:p.[Ala4Ser;Ala14Ser;Asp21Metfs*4]}
\end{frame}

\begin{frame}
  \frametitle{Protein descriptions (2)}

  Input:

  \bt{NM\_002001.2:n.[159del;162\_163del]}
  \bigskip
  \pause

  Old:

  \bt{NM\_002001.2:n.[159del;162\_163del]}

  \bt{NM\_002001.2:p.?}
  \bigskip
  \pause

  New:

  \bt{NM\_002001.2:n.[159del;162\_163del]}

  \bt{NM\_002001.2:p.Asp21\_Val22delinsSer}
\end{frame}

\begin{frame}
  \frametitle{Combining variants}

  Input ($110$ and $111$ have the same nucleotide):

  \bt{NM\_002001.2:n.[109del;111del]}
  \bigskip
  \pause

  Old:

  \bt{NM\_002001.2:n.[109del;111del]}

  \bt{NM\_002001.2:p.?}
  \bigskip
  \pause

  New:

  \bt{NM\_002001.2:n.109\_110del}

  \bt{NM\_002001.2:p.Ala4Hisfs*27}
\end{frame}

\begin{frame}
  \frametitle{Splitting variants}

  Input:

  \bt{NM\_002001.2:c.40\_50delinsTCCTTACTGTG}
  \bigskip
  \pause

  Old:

  \bt{NM\_002001.2:n.139\_149delinsTCCTTACTGTG}

  \bt{NM\_002001.2:p.Ala14\_Phe17delinsSerLeuLeuCys}
  \bigskip
  \pause

  New:

  \bt{NM\_002001.2:n.[139G>T;149T>G]}

  \bt{NM\_002001.2:p.[Ala14Ser;Phe17Cys]}
\end{frame}

\begin{frame}
  \frametitle{Comparing reference sequences}

  DMD Dp71ab vs. DMD Dp71b:
  \bigskip

  Input:

  \bt{NM\_004018.2} and \bt{NM\_004016.2}
  \bigskip
  \pause

  Output:

  \bt{1097\_1098insTCCCGTTACTCTGATCAACTTCTGGCCAGT\ldots}
  \bigskip

  Interpretation:

  This is an exon not present in Dp71ab.
\end{frame}

\begin{frame}
  \frametitle{Old vs. new transcripts}

  DMD Dp71ab old vs. new:
  \bigskip

  Input: \bt{NM\_004018.2} and \bt{NM\_004018.1}
  \bigskip

  Output: \bt{[3308A>G;4288A>G]}
  \bigskip
  \bigskip
  \bigskip
  \pause

  FCER1A old vs. new:
  \bigskip

  Input: \bt{NM\_002001.1} and \bt{NM\_002001.2}
  \bigskip

  Output: \bt{1\_7del}
\end{frame}


\begin{frame}
  \frametitle{Old vs. new transcripts (2)}

  FCER2 old vs. new:
  \bigskip

  Input:

  \bt{NM\_002002.1} and \bt{NM\_002002.4}
  \bigskip
  \pause

  Output:

  \bt{[720C>T;903A>G;930T>C;1019C>A; \\
    1401\_1402insACACCCCAACAGCACCCTCTCCAGATGAGAGT\ldots; \\
    1478del;1529\_1530insTCCCACATTTGTCCCCTTCTTGGA\ldots]}
  \smallskip
  \pause

  vice versa:

  \bt{[720T>C;903G>A;930C>T;1019A>C;1402\_1464del; \\
    1540dup;1592\_1620del]}
\end{frame}

\begin{frame}
  \frametitle{Limitations}

  mtDNA reference vs. isolate K422 mitochondrion
  \medskip

  Input: \bt{NC\_012920.1} and \bt{JX266268.1}
  \medskip
  \pause

  Output:
  \bt{
    [73A>G;194C>T;249del;263A>G;310delinsCTC;489T>C; \\
    750A>G;1438A>G;1715C>T;2231\_2232dup;2706A>G; \\
    3107del;3552T>A;4715A>G;4769A>G;6026G>A;7028C>T; \\
    7196C>A;7999T>C;8508A>G;8584G>A;8701A>G;8860A>G; \\
    9540T>C;9545A>G;10398\_10400delinsGCT;10873T>C; \\
    11719G>A;11914G>A;11969G>A;12672A>G;12705C>T; \\
    13263A>G;14318T>C;14766C>T;14783T>C;15043G>A; \\
    15204T>C;15301G>A;15326A>G;15487A>T;15968T>C; \\
    16129G>A;16223C>T;16298T>C;16327C>T;16519T>C]
  }
  \medskip
  \pause

  Runtime: $\pm20$ minutes, Memory: $4$G.
\end{frame}

\section{Optimisation}
\begin{frame}
  \frametitle{Accuracy vs. speed}

  \begin{tabular}{l@{\ \ $\Rightarrow$\ \ }l}
    \bt{AGAGGACG} & \bt{AG AG GA CG} \\
    \bt{GAGGACA}  & \bt{GA AG GG GA AC CA}
  \end{tabular}
  \pause

  \begin{table}
    \begin{center}
      \begin{tabular}{l|llll}
           & \bt{A} & \bt{A} & \bt{G} & \bt{C} \\
           & \bt{G} & \bt{G} & \bt{A} & \bt{G} \\
        \hline
        \bt{GA} & 0 & 0 & 1 & 0 \\
        \bt{AG} & 1 & \onslide<3>{\color{red}}1 & 0 & 0 \\
        \bt{GG} & 0 & 0 & 0 & 0 \\
        \bt{GA} & 0 & 0 & \onslide<3>{\color{red}}2 & 0 \\
        \bt{AC} & 0 & 0 & 0 & 0 \\
        \bt{CA} & 0 & 0 & 0 & 0 \\
      \end{tabular}
    \end{center}
    \caption{Rough method to find large strings.}
  \end{table}

  \onslide<3>{We make a ``knight move''.}
\end{frame}

\begin{frame}
  \frametitle{Accuracy vs. speed(2)}

  \begin{minipage}[t]{0.45\textwidth}
    \begin{table}[]
      \begin{center}
        \begin{tabular}{l|llll}
             & \bt{A} & \bt{A} & \bt{G} & \bt{C} \\
             & \bt{G} & \bt{G} & \bt{A} & \bt{G} \\
          \hline
          \bt{GA} & 0 & 0 & 1 & 0 \\
          \bt{AG} & 1 & 1 & 0 & 0 \\
          \bt{GG} & 0 & 0 & 0 & 0 \\
          \bt{GA} & 0 & 0 & 2 & 0 \\
          \bt{AC} & 0 & 0 & 0 & 0 \\
          \bt{CA} & 0 & 0 & 0 & 0 \\
        \end{tabular}
      \end{center}
      \caption{``Zoom out'' $k = 2$.}
    \end{table}
  \end{minipage}
  \hfill
  \begin{minipage}[t]{0.45\textwidth}
    \begin{table}[]
      \begin{center}
        \begin{tabular}{l|ll}
              & \bt{A} & \bt{G} \\
              & \bt{G} & \bt{G} \\
              & \bt{A} & \bt{A} \\
          \hline
          \bt{GAG} & 0 & 0 \\
          \bt{AGG} & 0 & 0 \\
          \bt{GGA} & 0 & 1 \\
          \bt{GAC} & 0 & 0 \\
          \bt{ACA} & 0 & 0 \\
        \end{tabular}
      \end{center}
      \caption{``Zoom out'' $k = 3$.}
    \end{table}
  \end{minipage}
  \pause

  We find all common sub strings larger than $k$.
  \pause

  The length of these strings are at least $\ell k$ and at most
   $\ell k + (k - 1)$ long.
\end{frame}

\section{Conclusions}
\begin{frame}
  \frametitle{We are getting there}

  Extracting descriptions is feasible.
  \pause
  \begin{itemize}
    \item Guarantees the same description for the same variant, no matter how
      it is described by the user.
    \item Usable for comparing reference sequences.
    \pause
    \begin{itemize}
      \item Real lift over.
    \end{itemize}
  \end{itemize}
  \bigskip
  \pause

  Extracting descriptions is practical.
  \begin{itemize}
    \item By ``zooming out'', we can meet the memory requirements.
    \begin{itemize}
      \item $4$G to less than a megabyte.
    \end{itemize}
    \item By ``zooming out'', we can meet the processing requirements.
    \begin{itemize}
      \item mtDNA test: $20$ minutes to under one second.
    \end{itemize}
  \end{itemize}
\end{frame}

\section{Questions?}
\lastpagetemplate
\begin{frame}
  \begin{center}
    Acknowledgements:
    \bigskip
    \bigskip

    Martijn Vermaat

    Ivo Fokkema

    Peter Taschner

    Johan den Dunnen

  \end{center}
\end{frame}

\end{document}
