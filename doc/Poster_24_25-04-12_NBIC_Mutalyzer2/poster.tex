\documentclass[final, slidestop]{beamer}

\title{Generating variant descriptions based on sequence comparison}
\author{Jeroen F.J. Laros, Martijn Vermaat, Johan T. den Dunnen,
  Peter E.M. Taschner}
\institute{Center for Human and Clinical Genetics, Leiden University Medical
  Center, Leiden, The Netherlands}
\providecommand{\centerLogo}{
  \includegraphics[height = 3cm]{gen2phen_logo}
}
\providecommand{\rightLogo}{
  \includegraphics[height = 3cm]{nbic_logo}
}
\providecommand{\colOneWidth}{0.48}
\providecommand{\colTwoWidth}{0.48}

\usetheme{lumc}

\begin{document}

\begin{frame}{}
  \begin{myPoster}
    \colorBlock{Background}{Introduction}{1}{
      Unambiguous and correct sequence variant descriptions are of utmost
      importance for DNA diagnostics. Descriptions can be checked and corrected
      with the {\em Mutalyzer sequence variation nomenclature checker}
      (Fig.~\ref{figure:namecheck}) (\color{red}\bt{http://www.mutalyzer.nl/}\color{LUMCBlue})
      following the Human Genome Variation Society (HGVS) sequence variant
      nomenclature recommendations~\cite{HGVS}.

      \vspace{1cm}

      Initial construction of variant descriptions, however, requires comparison
      of the reference sequence and the variant sequence and basic knowledge of
      the HGVS recommendations. With the arrival of long read sequencers (e.g.
      PacBio) and the rise of sophisticated variant callers, the chance of
      finding a complex variant increases and so does the need to describe these
      variants.
    }
    \colorBlock{SalmonBg}{Description Extraction}{1}{
      The description extractor can be used to construct variant descriptions
      according to the HGVS recommendations. The algorithm closely follows the
      human approach to describe a variant. It will first find the ``area of
      change'' and then finds the largest overlap between the original area
      and the area in the observed sequence. This process is repeated until
      the smallest description is found. This not only helps clinicians to
      generate the correct description, but its implementation also allows
      automation of the description process.

      \vspace{1cm}

      As an example, consider the following two DNA sequences:

      \vspace{1cm}

      \centerline{\texttt{ATGATGATCAGATACAGTGTGATACAGGTAGTTAGACAA}}
      \centerline{\texttt{ATGATTTGATCAGATACATGTGATACCGGTAGTTAGGACAA}}

      \vspace{1cm}

      By comparing the sequences, the following HGVS description will be
      extracted: \texttt{g.[5\_6insTT;17del;26A>C;35dup]}

      \vspace{1cm}

      \begin{table}
        \caption{Overview of the raw variants as provided by the description
          extractor.}
        \colorbox{white}{
        {\small
          \begin{tabular}{l|l|l|l|l|l|l}
            Start & End & Type  & Deleted    & Inserted    & Shift & Description \\
            \hline
            5     & 6   & ins   &            & \texttt{TT} & 1     & \texttt{5\_6insTT} \\
            17    & 0   & del   &            &             & 0     & \texttt{17del} \\
            26    & 0   & subst & \texttt{A} & \texttt{C}  & 0     & \texttt{26A>C} \\
            35    & 0   & dup   &            &             & 1     & \texttt{35dup} \\
          \end{tabular}
          }
        }
      \end{table}
    }
    \colorBlock{GreenBg}{Mutalyzer Interfaces}{1}{
      \begin{tabular}{l@{\ \ --\ \ }p{25cm}}
        Name Checker       & Syntactic and semantic checks.$^*$
                               (Fig.~\ref{figure:namecheck}) \\
        Syntax Checker     & Syntactic checks only.$^*$ \\
        Position Converter & Convert chromosomal positions to gene-centered
                               notation (no semantic check).$^*$ \\
        SNP Converter      & Convert a dbSNP rsId to HGVS notation.$^*$ \\
        Name Generator     & Contruct a HGVS notation. \\
        Description Extractor & Extract HGVS notation from sequences. \\
        GenBank Uploader   & Upload custom GenBank files. \\
        Webservices        & Programmatic (SOAP) interface. \\
      \end{tabular}
      \bigskip

      $^*$ {\small Also available as a batch interface.}
    }
    \colorBlock{YellowBg}{Acknowledgements}{1}{
      {\small
        Funded by the European Community's Seventh Framework Programme
        (FP7/2007-2013) under grant agreement no. 200754 - the GEN2PHEN project.
      }

    }
    \nextColumn
    \colorBlock{BlueBg}{Name Checking}{1}{
      \begin{figure}
        {
          \includegraphics[width = 0.95\textwidth, height = 57cm]{mutalyzerNameCheck}
        }
        \caption{Mutalyzer 2.0 Name Checker results using the CDKN2A LRG
          reference sequence~\cite{LRG}}
        \label{figure:namecheck}
      \end{figure}
    }
    \colorBlock{WhiteBg}{Conclusions}{1}{
      Variants at intergenic, exonic, intronic, CDS and UTR positions can be
      easily distinguished based on their gene-centered HGVS descriptions.
      Mutalyzer facilitates batch-wise conversion from dbSNP rsIDs or
      chromosomal position numbering of next generation sequencing data to
      transcript position numbering, as well as sequence variant checking of
      locus-specific sequence variant databases (LSDBs)~\cite{LOVD}.
    }
    \colorBlock{Background}{References}{1}{
      {\small
        \bibliography{$HOME/projects/bibliography}{}
      }
    }
  \end{myPoster}
\end{frame}
\end{document}
